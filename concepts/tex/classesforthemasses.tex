% This is "sig-alternate.tex" V2.1 April 2013
% This file should be compiled with V2.5 of "sig-alternate.cls" May 2012
%
% This example file demonstrates the use of the 'sig-alternate.cls'
% V2.5 LaTeX2e document class file. It is for those submitting
% articles to ACM Conference Proceedings WHO DO NOT WISH TO
% STRICTLY ADHERE TO THE SIGS (PUBS-BOARD-ENDORSED) STYLE.
% The 'sig-alternate.cls' file will produce a similar-looking,
% albeit, 'tighter' paper resulting in, invariably, fewer pages.
%
% ----------------------------------------------------------------------------------------------------------------
% This .tex file (and associated .cls V2.5) produces:
%       1) The Permission Statement
%       2) The Conference (location) Info information
%       3) The Copyright Line with ACM data
%       4) NO page numbers
%
% as against the acm_proc_article-sp.cls file which
% DOES NOT produce 1) thru' 3) above.
%
% Using 'sig-alternate.cls' you have control, however, from within
% the source .tex file, over both the CopyrightYear
% (defaulted to 200X) and the ACM Copyright Data
% (defaulted to X-XXXXX-XX-X/XX/XX).
% e.g.
% \CopyrightYear{2007} will cause 2007 to appear in the copyright line.
% \crdata{0-12345-67-8/90/12} will cause 0-12345-67-8/90/12 to appear in the copyright line.
%
% ---------------------------------------------------------------------------------------------------------------
% This .tex source is an example which *does* use
% the .bib file (from which the .bbl file % is produced).
% REMEMBER HOWEVER: After having produced the .bbl file,
% and prior to final submission, you *NEED* to 'insert'
% your .bbl file into your source .tex file so as to provide
% ONE 'self-contained' source file.
%
% ================= IF YOU HAVE QUESTIONS =======================
% Questions regarding the SIGS styles, SIGS policies and
% procedures, Conferences etc. should be sent to
% Adrienne Griscti (griscti@acm.org)
%
% Technical questions _only_ to
% Gerald Murray (murray@hq.acm.org)
% ===============================================================
%
% For tracking purposes - this is V2.0 - May 2012

\documentclass[preprint]{sig-alternate-05-2015}




%\usepackage{color}
\usepackage{url}
\usepackage[usenames,dvipsnames,svgnames,table]{xcolor}
\usepackage{listings}
\definecolor{StringRed}{rgb}{.637,0.082,0.082}
\definecolor{CommentGreen}{rgb}{0,0.5,0}
\definecolor{KeywordBlue}{rgb}{0,0,1}
\definecolor{UserTypeTeal}{rgb}{0.1679,0.5664,0.6836}

\lstdefinelanguage{threeextensions}%
  {morekeywords={var,from,select,where,orderby,yield,let,join,on,into,equals,group,by,ascending,descending,
                 try,catch,finally},
   sensitive,
  }[keywords]%

\lstdefinelanguage{fourextensions}
   {morekeywords={dynamic},
    sensitive
   }[keywords]

\lstdefinelanguage{fiveextensions}
   {morekeywords={await,async},
    sensitive
   }[keywords]

\lstdefinelanguage{CSharpLatest}
                         {language=[Sharp]{C}, %
                          alsolanguage=threeextensions,
                          alsolanguage=fourextensions,
					alsolanguage=fiveextensions,
                          basicstyle=\color{black}\ttfamily,
                          showstringspaces=false,
                          keywordstyle=\color{KeywordBlue},
                          commentstyle=\color{CommentGreen},
                          stringstyle=\color{StringRed},
rangebeginprefix=\#region\ ,
rangeendprefix=\#endregion\ ,
includerangemarker=false,
                          columns=flexible
                         }

\lstdefinelanguage{conceptextensions}
   {morekeywords={instance,concept},
    sensitive
   }[keywords]
\lstdefinelanguage{FSharp}
                         {language={Caml}, %
                          alsolanguage=conceptextensions,
                          basicstyle=\color{black}\ttfamily,
                          showstringspaces=false,
                          keywordstyle=\color{KeywordBlue},
                          commentstyle=\color{CommentGreen},
                          stringstyle=\color{StringRed},
                          columns=flexible
                         }


\lstset{moredelim=[is][\color{UserTypeTeal}]{|}{|}}


\begin{document}

\setcopyright{none}
% Copyright
%\setcopyright{acmcopyright}
%\setcopyright{acmlicensed}
%\setcopyright{rightsretained}
%\setcopyright{usgov}
%\setcopyright{usgovmixed}
%\setcopyright{cagov}
%\setcopyright{cagovmixed}

%
% --- Author Metadata here ---
%\conferenceinfo{ML Workshop 2016}{Japan}
%\CopyrightYear{2007} % Allows default copyright year (20XX) to be over-ridden - IF NEED BE.
%\crdata{0-12345-67-8/90/01}  % Allows default copyright data (0-89791-88-6/97/05) to be over-ridden - IF NEED BE.
% --- End of Author Metadata ---

\title{Classes for the Masses (Extended Abstract)}
%\titlenote{A full version of this paper is available as
%\textit{Author's Guide to Preparing ACM SIG Proceedings Using
%\LaTeX$2_\epsilon$\ and BibTeX} at
%\texttt{www.acm.org/eaddress.htm}}}
%
% You need the command \numberofauthors to handle the 'placement
% and alignment' of the authors beneath the title.
%
% For aesthetic reasons, we recommend 'three authors at a time'
% i.e. three 'name/affiliation blocks' be placed beneath the title.
%
% NOTE: You are NOT restricted in how many 'rows' of
% "name/affiliations" may appear. We just ask that you restrict
% the number of 'columns' to three.
%
% Because of the available 'opening page real-estate'
% we ask you to refrain from putting more than six authors
% (two rows with three columns) beneath the article title.
% More than six makes the first-page appear very cluttered indeed.
%
% Use the \alignauthor commands to handle the names
% and affiliations for an 'aesthetic maximum' of six authors.
% Add names, affiliations, addresses for
% the seventh etc. author(s) as the argument for the
% \additionalauthors command.
% These 'additional authors' will be output/set for you
% without further effort on your part as the last section in
% the body of your article BEFORE References or any Appendices.

\numberofauthors{2} %  in this sample file, there are a *total*
% of EIGHT authors. SIX appear on the 'first-page' (for formatting
% reasons) and the remaining two appear in the \additionalauthors section.
%
\author{
% You can go ahead and credit any number of authors here,
% e.g. one 'row of three' or two rows (consisting of one row of three
% and a second row of one, two or three).
%
% The command \alignauthor (no curly braces needed) should
% precede each author name, affiliation/snail-mail address and
% e-mail address. Additionally, tag each line of
% affiliation/address with \affaddr, and tag the
% e-mail address with \email.
%
% 1st. author
\alignauthor
Claudio Russo \affaddr{(Microsoft Research)}\\
% 2nd. author
\alignauthor
Matt Windsor \affaddr{(University of York)}\\
}


\makeatletter
\def\@copyrightspace{\relax}
\makeatother



\maketitle
\begin{abstract}
Type classes are an immensely popular and productive feature of Haskell. They have since been adopted in, and adapted to, numerous other languages, including theorem provers.
This talk will sketch that type classes have a natural and efficient representation in .NET. 
This paves the way for the extension of F\# and other .NET languages with Haskell style type classes. 
The representation is type preserving and promises easy and safe cross-language inter-operation.
We are currently, and rapidly, extending the open source C\# compiler and language service, Roslyn, with support for type classes but intend to
do the same for F\# once that work has been completed.
\end{abstract}


% We no longer use \terms command
%\terms{Theory}



\section{Introduction}


Haskell's \emph{type classes} \cite{peytonjones:h98,WadlerBlott} are a powerful abstraction mechanism for describing generic algorithms applicable to types that have different representations but common interfaces.
A \emph{type class} is a predicate on types that specifies a set of required operations by their type signatures.
A type may be declared to be an \emph{instance} of a type class, and must supply an implementation for each of the class' operations. 
Type classes may be arranged hierarchically, permitting subsumption and inheritance. A default method may also be associated with an operation allowing its implementations to be omitted.

Many modern language have adopted features inspired
by type classes, with different implementation techniques.  Scala has
\emph{implicits}\cite{scalaimplicits}, implicit method arguments denoting dictionaries, that
are inferred by the compiler but represented, at run-time, as additional
heap-allocated arguments to methods (with commensurate overhead).  C++ came very
close to adopting \emph{concepts} \cite{cppconcepts}, a rather different compile-time template mechanism, 
directly inspired by Haskell's type classes but emphasize compile-time code specialization. Rust has \emph{traits} \cite{Rust} and Swift has
\emph{protocols} \cite{Swift}.

{\bf Contribution} We describe a simple encoding that allows us to add type classes to
any .NET language, allowing interoperable definitions of type
classes. Our encoding relies on the CLR's distinctive approach to
representing and compiling generic code\cite{genericspldi,genericspopl}. Unlike, for example, the JVM, the CLR
byte-code format is fully generic (all source level type information,
including class and method type parameters, are represented in the
metadata and virtual instruction set).  Parameterized code is JIT-compiled to type passing code,
with type parameters having run-time representations as (second-order)
values.  The JIT compiler uses the reified types to generate
specialized memory representations (for instantiated generic types)
and specialized (and thus more efficient) code for generic
methods. For example, scalar types and compounds of scalars called
structs have natural unboxed representations familiar to C(++)
programmers; generic array manipulating code will manipulate array
elements without boxing when instantiated at scalar types.  This run-time
specialization allows the JIT to avoid the uniform (i.e.\ lowest-common-denominator) representations adopted by many implementations of ML, 
Haskell, the JVM and most dynamic languages.

Haskell compilers typically compile type classes using the so-called
\emph{dictionary translation}.  The translation, guided by source
types, inserts terms that justify type class constraints. These terms,
or witnesses, are records of functions that provide implementations
for all of the constraint's methods. Although similar to
object-oriented virtual method tables, witness are not attached to
objects, but passed separately as function arguments.  Because type
classes are resolved statically, aggressive inlining can remove most,
but not all, indirection through witness parameters. This leads to
efficient code with few indirect calls and leaner representations of values than full-blown objects. Objects, in contrast, must lug their method-tables wherever they go.

Given the obvious similarity between type passing and dictionary
passing, it is perhaps not surprising that type passing forms an
excellent implementation technique for Haskell's dictionary passing. 
This talk will give on overview of the technique that we are applying to provide
efficient, interoperable type class implementations to both C\# and F\#.


\section{The Representation}

This section sketches our representation of the Haskell'98 type classes on .NET by example.
For each example, we give the Haskell code, underlying .NET code in vanilla C\#, and proposed F\# syntax.
We use vanilla C\# as a more readable proxy for .NET intermediate bytecode and metadata.

\subsection{Haskell Type Classes}

A Haskell type class, for example:
\lstset{language={CSharpLatest}}

{
\lstset{language={Haskell}}
\begin{lstlisting}
  class Eq a where 
    (==) :: a -> a -> Bool
\end{lstlisting}
}
Is naturally represented in C\# as:
\lstset{language={CSharpLatest}}
\begin{lstlisting}
  interface Eq<A> {
    bool Equals(A a, A b);
  }
\end{lstlisting}

For new F\# syntax we adopt the keyword \lstinline{concept} (\lstinline{class} is already taken):

\lstset{language={FSharp}}
\begin{lstlisting}
concept Eq 'a where 
    (==): 'a -> 'a -> bool
\end{lstlisting}
 
\subsection{Haskell Overloads}

The Haskell declaration of class \lstinline{Eq a} implicitly declares the overloaded 
operations induced by class' members.

\lstset{language={Haskell}}
\begin{lstlisting}
    (==) :: (Eq a) => a -> a -> Bool 
\end{lstlisting}
Observe that the overload operation has a more general constrained type \lstinline{(Eq q) => ...}.

This generic operation can be represent in C\# by the method:
\lstset{language={CSharpLatest}}
\begin{lstlisting}
   public static bool Equals<A,EqA>(A a, A b) 
      where EqA : struct, Eq<A> {
        return default(EqA).Equals(a, b); }
\end{lstlisting}
This method has not one, but \emph{two}, type parameters. The first, \lstinline{A}, is just the type parameter from the declaration.
The second, \lstinline{EqA}, is a type parameter that is constrained to be a struct and witnesses the constraint that \lstinline{A} supports interface \lstinline{Eq<A>}.

The use of the \lstinline{struct} constraint on \lstinline{EqA} is significant and subtle. Structs are stack allocated so essentially free to create.
Moreover, every struct, including type parameters \lstinline{T} of kind struct, has a default (all-zero) value that is created by 
calling \lstinline{default(T)}.

Invoking a method on a default value of reference type would simply raise a null-reference exception because the receiver is \lstinline{null}.
However, methods on structs (including interface methods) can always be properly invoked
by calling the method on the struct's default value. 

Thus an operation over some class can be represented as a static generic method, 
parameterized by an additional dictionary type parameter (here \lstinline{EqA}).
Similarly, derived operations with type class constraints can be represented by generic methods with suitably constrained type parameters.

So a Haskell dictionary \emph{value} corresponds to a  C\# dictionary \emph{type}.

%The dictionary type parameter is marked \lstinline{struct} (so stack allocated):
%we can access its operations through a default value (no need to pass dictionary values).


\subsection{Haskell Instances}

A Haskell instances declarations is represented by the declaration of an empty (field-less) .NET generic struct that implements the associated type class (itself an interface).
This gives us a cheap representation of Haskell instances. 
%A Haskell instance declaration can be represented by the declaration of a C\# struct type.
%Moreover, the dictionary parameter corresponding to a Haskell type constraint can just be represented by a C\# type parameter; in particular, by a type parameter that is both a struct and implements the corresponding C\# interface.

For example, the Haskell instance declaration:
\lstset{language={Haskell}}
\begin{lstlisting}
instance Eq Integer where 
  x == y  =  x `integerEq` y
\end{lstlisting}

can be represented by the C\# structure:
\lstset{language={CSharpLatest}}
\begin{lstlisting}
struct EqInt : Eq<int>  {
  public bool Equals(int a, int b) { return a == b; }  
}
\end{lstlisting}

For F\#, we introduce the shorter \lstinline{instance} declaration:

\lstset{language={FSharp}}
\begin{lstlisting}
instance Eq int where 
  Equals (a, b) = a = b
\end{lstlisting}

Note that the F\# syntax, like Haskell, elides the name of the instance in the more explicit C\# representation.
In Haskell, instances are anonymous.


\subsection{Derived Instances}

This Haskell code defines, given an equality on (any) type \lstinline{a}'s, an equality operation on lists of a.
\lstset{language={Haskell}}
\begin{lstlisting}
instance (Eq a) => Eq ([a]) where 
  nil    == nil    = true
  (a:as) == (b:bs) =  (a == b) && (as == bs)
  _     == _     = false
\end{lstlisting}


We can represent such a Haskell parameterized instance as a \emph{generic} struct:

%\lstset{language={CSharpLatest}}
%\begin{lstlisting}
%struct EqList<A, EqA> : Eq<List<A>>
%       where EqA : struct, Eq<A>  {
%  public bool Equals(List<A> a, List<A> b) {
%    if (a.IsNull != b.IsNull) return true;
%    else if (a.IsCons & b.IsCons) 
%         return default(EqA).Equals(a.Head,b.Head) 
%                && Equals(a.Tail,b.Tail);
%         else false;
%  }}
%\end{lstlisting}

\lstset{language={CSharpLatest}}
\begin{lstlisting}
struct EqList<A, EqA> : Eq<List<A>>
       where EqA : struct, Eq<A>  {
  public bool Equals(List<A> a, List<A> b) {
    return     (a.IsNull && b.IsNull)
              ||  (a.IsCons && b.IsCons 
                    && default(EqA).Equals(a.Head,b.Head) 
                    && Equals(a.Tail,b.Tail));
  }}
\end{lstlisting}

This struct implements the interface \lstinline{Eq<List<A>>}, but only when instantiated with a suitable type argument and witness for constraint \lstinline{Eq<A>}.
Notice that \lstinline{EqList} has, once again, an additional type parameter witnessing \lstinline{Eq<A>}.
Instantiations of \lstinline{EqList} constuct witnesses for \lstinline{Eq<List<A>>}.

For F\# we adopt the mode concise, nameless declaration:

\lstset{language={FSharp}}
\begin{lstlisting}
instance Eq 'a => Eq (List 'a) where 
  Equals(a,b) = match a,b with
    | [],[] -> true
    | (a::l),(b::m) -> Equals(a,b) && Equals(l,m)
    | _,_ -> false
\end{lstlisting}

\subsection{Other features}

We do not have space to describe the representations of other features but suffice to say that we can encode:
Haskell type class operations that themselves have constrained types in their signatures (using interface methods that are generic);
Haskell type class hierarchies  using interface inheritance; default operations using shared static methods; instances requiring 
polymorphic recursion; instances as data (to constrained term constructors) and multi-parameter type classes. Moreover, choosing to provide named witness rather than anonymous one allows us to support
explicit as well as implicit instances.
The only feature we cannot currently support is higher-kinded type classes (like \lstinline{Monad}), but this is because .NET does not yet support higher kinds.

Much of the concision of Haskell comes not only from the declaration of type class hierarchies but from the implicit solution of constraints during type inference.
For C\#, we envision witness inference to be a mild generalization of type argument inference, with instantiations derived from the pervasive and locally assumed concept hierarchy.
For F\#, we hope to adapt Haskell's more elaborate techniques for not only solving but also implicitly propagating inferred type class constraints.
\section{Implementation}

We are currently proiritizing our efforts on our C\# implementation of type classes \cite{Roslynfork}, forked from Microsoft's open source Roslyn compiler \cite{Roslyn}, adopting a syntax loosely inspired by C++ concept proposal 
%This is primarily because of interest from the C\# team, because C\# still has wider reach than F\# and because C\# is arguably in greater need of abstraction facilities.
However, we hope to turn our attention to F\# over the second half of the summer.

\bibliographystyle{abbrv}
\bibliography{sigproc} 
\end{document}
